\chapter{Conclusion}
\label{chap:conclusion}

This chapter concludes the research by synthesizing its overall contributions, answering the guiding research questions, and highlighting the broader significance of the findings. While the preceding chapters presented the methodological design, results, and detailed discussion, this chapter provides a concise summary of achievements, reflects on their implications, and offers closing remarks on the value and future of hybrid retrieval for emotion recognition with large language models (LLMs).

\section{Summary of the Research}
The study set out to address the instability and inconsistency of few-shot emotion classification in LLMs. To do so, it applied the Design Science Research (DSR) methodology to design, build, and evaluate a novel computational artifact: a hybrid retrieval and adaptive N-shot prompting pipeline.  

Iteration~1 focused on data engineering and produced a validated, reproducible benchmark dataset across three levels of granularity (27, 15, 3 labels), addressing the need for a reliable evaluation foundation. Iteration~2 introduced adaptive N-shot prompting, demonstrating that few-shot learning substantially improves stability and performance, but with rapidly diminishing returns beyond small $N^{\ast}$ values. Iteration~3 extended this by introducing retrieval-guided prompting, showing that the quality of demonstrations is as important as their quantity. Hybrid retrieval, which combines lexical and semantic signals, consistently outperformed all baselines across granularities.  

Together, these iterations advanced both methodological rigor and practical utility, showing that emotion classification can be made more stable, efficient, and interpretable when retrieval and adaptive prompting are combined.

\section{Contributions of the Study}
This research makes several key contributions. Methodologically, it introduces a structured pipeline that integrates dataset engineering, adaptive few-shot learning, and hybrid retrieval into a coherent workflow. Empirically, it provides evidence that both the quantity and quality of demonstrations strongly influence LLM performance in emotion recognition. Practically, it demonstrates a scenario-based application that allows evaluators to interactively test and validate results, bridging the gap between academic experimentation and real-world deployment.

\section{Implications}
The implications of this study extend across both academic and applied domains. For researchers, the validated dataset and adaptive prompting methodology provide a reproducible platform for further investigation. For practitioners, the hybrid retrieval approach offers a cost-efficient and interpretable strategy for deploying emotion-aware LLMs in sensitive applications such as customer support, mental health, and social media moderation. More broadly, the work contributes to the emerging field of emotionally intelligent AI, advancing the transparency and reliability of systems that interact with human affect.

\section{Closing Reflections}
While limitations and delimitations were acknowledged , this research has established a foundation for future work that can extend the pipeline to dynamic weighting, multi-dataset integration, and cross-platform benchmarking on services such as AWS Bedrock. The findings confirm that emotionally intelligent AI systems require both methodological rigor and practical adaptability.  

In conclusion, the study demonstrates that unstable and inconsistent performance in LLM-based emotion recognition can be addressed by combining adaptive N-shot prompting with hybrid retrieval. This contribution advances the state of the art in affective computing and provides actionable insights for both academia and industry. By ensuring that contextual information is not only sufficient in quantity but also relevant in quality, this research paves the way toward more reliable, transparent, and human-aligned applications of large language models.
\section{Future Work}
Building upon the foundations established in this study, several promising directions can be explored:
\begin{itemize}
    \item \textbf{Dynamic $\lambda$-Optimization:} Implement adaptive weighting between lexical and semantic retrieval based on context entropy.
    \item \textbf{Cross-Dataset Evaluation:} Extend the framework to additional corpora such as EmpatheticDialogues or EmotionStimulus.
    \item \textbf{Multimodal Emotion Fusion:} Combine textual emotion recognition with speech or facial cues for richer affective understanding.
    \item \textbf{Real-Time User Studies:} Conduct longitudinal evaluations within mental health and education applications.
    \item \textbf{Explainability Toolkit:} Integrate model-agnostic XAI techniques (e.g., SHAP, LIME) into the Streamlit dashboard for interactive transparency.
\end{itemize}
These future extensions would enhance the artifact’s generalization, interpretability, and ethical resilience, bringing emotionally intelligent AI closer to human-centered interaction.
