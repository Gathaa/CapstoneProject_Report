% ==============================================================================
% APPENDIX F: Example N-Shot Prompt
% File: appendices/appendix_f_prompt.tex
% ==============================================================================

\chapter{Appendix F: Example N-Shot Prompt Template}
\label{app:prompt}

This appendix provides an example of the prompt structure used to query the GPT-4 model in Iteration 3 and 4 for the N-shot classification task. The prompt is designed to be clear, unambiguous, and structured to elicit a consistent JSON output.

\section{Prompt Architecture}

The prompt consists of three main components:
\begin{enumerate}
    \item \textbf{The System Instruction:} A high-level directive that establishes the model's role, the task, the list of possible labels, and the required output format.
    \item \textbf{The In-Context Examples (Few-Shot Demonstrations):} A balanced set of $N^*$ examples for each of the possible emotion labels.
    \item \textbf{The User Query:} The new, unseen text that needs to be classified.
\end{enumerate}

\section{Example Prompt for 3-Label Granularity (N* = 4)}

Below is a truncated example of a complete prompt that would be sent to the GPT-4 API for the 3-label classification task, where the adaptive mechanism determined that $N^*=4$.

\hrule
\vspace{1em}
\textbf{SYSTEM INSTRUCTION:}
\begin{verbatim}
You are an expert emotion classifier. Your task is to classify the user's
text into one of the following three categories: {Positive, Negative, Ambiguous}.

You must provide your answer in a JSON format with two keys:
1. "emotion": The single best emotion label from the list.
2. "confidence": A score from 0.0 to 1.0 indicating your confidence.

Here are some examples of how to classify text:

--- EXAMPLES ---

Text: "What a beautiful and wonderful day!"
{"emotion": "Positive", "confidence": 1.0}

Text: "I am so happy about my new promotion."
{"emotion": "Positive", "confidence": 1.0}

Text: "This is the best news I've heard all week."
{"emotion": "Positive", "confidence": 1.0}

Text: "Congratulations on your success, you deserve it!"
{"emotion": "Positive", "confidence": 1.0}

Text: "This is completely unacceptable service!"
{"emotion": "Negative", "confidence": 1.0}

Text: "I am furious with how I was treated."
{"emotion": "Negative", "confidence": 1.0}

Text: "The ending of that movie was just heartbreaking."
{"emotion": "Negative", "confidence": 1.0}

Text: "I feel so lonely and lost."
{"emotion": "Negative", "confidence": 1.0}

Text: "I'm not sure what to think about this new policy."
{"emotion": "Ambiguous", "confidence": 0.9}

Text: "That's an interesting perspective, I hadn't considered it."
{"emotion": "Ambiguous", "confidence": 0.8}

Text: "Hmm, I wonder what happens next."
{"emotion": "Ambiguous", "confidence": 0.9}

Text: "The results of the experiment are inconclusive."
{"emotion": "Ambiguous", "confidence": 1.0}

--- END OF EXAMPLES ---

Now, classify the following user text.
\end{verbatim}

\vspace{1em}
\hrule
\vspace{1em}
\textbf{USER QUERY:}
\begin{verbatim}
My flight was cancelled and the airline lost my luggage.
\end{verbatim}
\vspace{1em}
\hrule
\vspace{1em}

\section{Expected Model Output}

Given the prompt above, the expected output from the GPT-4 model would be a clean JSON object:
\begin{verbatim}
{
  "emotion": "Negative",
  "confidence": 1.0
}
\end{verbatim}

This structured approach ensures that the model's output is consistent and can be reliably parsed by the evaluation script, minimizing errors and ensuring reproducibility.